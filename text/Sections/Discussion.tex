From all the data that was presented, we can clearly say that there is a correlation between word length and the logarithm of the frequency rank and the logarithm of the relative frequency.

The $l \sim \log i$ correlation is positive except for the case of Chinese, which has mixed results depending on how word length is considered. For the case of the $l \sim -\log p$ correlation, we found it to be negative except for the case of Chinese, which, again, has mixed results.

Due to the nature of Chinese, it is challenging to analyze the language as it is unclear which measure of word length would be the best equivalent to those of Latin languages. For this reason, we believe it would be interesting to analyze Chinese with word duration and see if that provides more similar results to the rest of the languages.

For all of the languages, it was very surprising to see such consistent results with such small p-values in all the analyses, so it seems to have a logarithmic correlation which indicates an exponential relationship between word length and word frequency.

Although the language set that was studied is limited, it does offer a diverse representation of languages in terms of families and writing systems. We believe that the size of the analyzed dataset is optimal to see the results in detail without having too much, and aims to be a step into a larger analysis with a lot more languages which would go less in detail. The code used for this analysis can easily be used to analyze all the languages from the PUD and CV datasets.

We hope that this work will be useful for establishing global patterns in languages which can be useful for all types of language analysis.